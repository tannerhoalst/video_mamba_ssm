\documentclass{article}
\usepackage{graphicx} % Required for inserting images
\usepackage{float}

% Math packages
\usepackage{amsmath}        % for align, bmatrix, etc.
\usepackage{amsfonts}       % for \mathbb{}
\usepackage{amssymb}        % for extra math symbols

% For nicer text encoding (optional but recommended)
\usepackage[T1]{fontenc}
\usepackage[utf8]{inputenc}



\title{VideoMambaDepth}
\author{Tanner Hoalst}
\date{November 2025}

\begin{document}

\maketitle

\subsection*{Architecture}

Due to the lack of sufficient video depth data, we start with a pre-trained image depth estimation model,
Depth Anything V2, and adopt a joint training strategy using both image and video data.

\textbf{Depth Anything V2 Encoder.}
Depth Anything V2 is the current state-of-the-art monocular depth estimation model,
characterized by its high accuracy and generalization capabilities. We use its trained model as our encoder.
To reduce training costs and preserve well-learned features, the encoder is frozen during training.

Unlike monocular depth encoders that only accept image input, our training scenario requires the encoder to
process simultaneously both video and image data. To extract features from video frames with an image encoder,
we collapse the temporal dimension of a video clip into the batch dimension. The input data are denoted as
$\mathbf{X} \in \mathbb{R}^{(B \times N) \times C \times H \times W}$, where $B$ represents the batch size,
$N$ is the number of frames in the video clip ($N = 1$ for the image as input), and $C, H, W$ are the number
of channels, height, and width of the frames, respectively. The encoder takes $\mathbf{X}$ as input to
produce a series of intermediate feature maps
$\mathbf{F}_i \in \mathbb{R}^{(B \times N) \times (H/p \times W/p) \times C_i}$, where $p$ is the patch size
of the encoder. Although the image encoder extracts strong visual representations from individual frames, it
neglects the temporal information interactions between frames. Thus, the spatiotemporal head is introduced to
model the temporal relationships among the frames.

\textbf{Spatiotemporal Head.}
The spatiotemporal head (STH) is built upon the DPT head, with the only modification being the
insertion of temporal layers to capture temporal information. A temporal layer consists of a multi-head
self-attention model (SA) and a feed-forward network (FFN). When inputting a feature
$\mathbf{F}_i$ into the temporal layer, the temporal dimension $N$ is isolated, and self-attention is
executed solely along the temporal dimension to facilitate the interaction of temporal features. To capture
temporal positional relationships among different frames, we utilize absolute positional embedding to encode
temporal positional information from the video sequence.

The spatiotemporal head uniformly samples four feature maps from $\mathbf{F}_i$ (including the final features
from the encoder, denoted as $\mathbf{F}_4$) as inputs, and predicts a depth map
$\mathbf{D} \in \mathbb{R}^{H \times W}$. As shown in Figure~2, the selected features $\mathbf{F}_i$ are fed
into the Reassemble layer to produce a feature pyramid. Then, the features are gradually fused from low
resolution to high resolution by the Fusion layer. The Reassemble and Fusion layers are proposed by DPT.
 The final fused high-resolution feature maps are passed through the output layer to produce the
depth map $\mathbf{D}$. To reduce the additional computational load, we insert the temporal layer at a lower
stage, which deals with lower feature resolutions.



\subsection*{Temporal Gradient Matching Loss}

In this section, we start with the Optical Flow Based Warping (OPW) loss, then explore new loss designs and
ultimately propose a Temporal Gradient Matching Loss (TGM) that does not rely on optical flow, yet still
ensures the temporal consistency of predictions between frames.

\textbf{OPW loss.}
To constrain temporal consistency, previous video models such as assume that the depths at
corresponding positions in adjacent frames, identified through optical flow, are consistent, e.g., the Optical
Flow Based Warping (OPW) loss proposed in NVDIS. OPW loss is computed after obtaining corresponding
points on the basis of optical flow and warping. Specifically, for two consecutive depth prediction results,
$p_i$ and $p_{i+1}$, $p_{i+1}$ is warped to $\hat{p}_i$ according to the warping relationship derived from the
optical flow, and then the loss is calculated with:

\[
\mathcal{L}_{\mathrm{OPW}}
= \frac{1}{N-1} \sum_{i=1}^{N-1} \|\, p_i - \hat{p}_i \,\|_1,
\tag{1}
\]

where $N$ denotes the length of a video window, and $\|\cdot\|_1$ represents $\ell_1$ distance. However, there is
a fundamental issue with the OPW loss: the depth of corresponding points is \textit{not} invariant across adjacent
frames. This assumption holds true only when adjacent frames are stationary. For instance, in a driving scenario,
when a car is moving forward, the distance to static objects in front decreases relative to the car, violating
the assumption of $\mathcal{L}_{\mathrm{OPW}}$. To address this inherent issue of OPW, we propose a new loss
function to constrain the temporal consistency of depth.

\textbf{Temporal gradient matching loss (TGM).}
When calculating the loss, we do not assume that the depth of the corresponding points in adjacent frames remains
unchanged. Instead, we posit that the \textit{change} in depth of corresponding points between adjacent prediction
frames should be consistent with the change observed in ground truth. We refer to this discrepancy as a stable
error (SE) given by:

\[
\mathcal{L}_{\mathrm{SE}}
= \frac{1}{N-1} \sum_{i=1}^{N-1}
\big\|
\hat{d}_i - d_i
\;-\;
\big( \hat{g}_i - g_i \big)
\big\|_1,
\tag{2}
\]

Here, $d_i, g_i$ are scaled and shifted versions of the predictions and ground truth. $\hat{d}_i, \hat{g}_i$
denote the warped depth from the subsequent frame using optical flow. $|\cdot|$ is used to represent the absolute
values.

However, generating optical flow incurs additional overhead. To address the dependence on optical flow, we further
generalize the above assumption. Specifically, it is not necessary to use the corresponding points obtained from
optical flow. Instead, we directly use the depth at the same coordinate in adjacent frames to calculate the loss.
The assumption is that the change in depth at the same image position between adjacent frames should be consistent
with that in the ground truth. Since this process is akin to calculating the gradient of values in temporal
dimension, we name it Temporal Gradient Matching Loss, as given by:

\[
\mathcal{L}_{\mathrm{TGM}}
= \frac{1}{N-1} \sum_{i=1}^{N-1}
\big\|
d_{i+1} - d_i
\;-\;
\big( g_{i+1} - g_i \big)
\big\|_1.
\tag{3}
\]

In practice, we only compute the TGM loss in regions where the change in ground truth depth,
i.e., $|g_{i+1} - g_i| < 0.05$. This threshold helps to avoid sudden changes in depth map caused by edges,
dynamic objects, and other factors that introduce unsteadiness during training.

Our total loss to supervise video depth data is as follows:

\[
\mathcal{L}_{\mathrm{all}}
= \alpha \mathcal{L}_{\mathrm{TGM}} + \beta \mathcal{L}_{\mathrm{SE}},
\tag{4}
\]

where $\mathcal{L}_{\mathrm{SE}}$ is a scale- and shift-invariant loss to supervise single images proposed by
MiDaS. $\alpha$ and $\beta$ are weights to balance spatio-temporal consistency and spatial structure in a
single frame.



\section*{Inference strategy for super-long sequence}

\begin{figure}[t]
    \centering
    \includegraphics[width=\linewidth]{figures/inference_strategy_long_videos.png}
    \caption{\textbf{Inference strategy for long videos.} 
    $N$ is the video clip length consumed by our model. Each inference video clip is built by 
    $N - T_0 - T_k$ future frames, $T_0$ overlapping/adjacent frames, and $T_k$ key frames. 
    The key frames are selected by taking every $\Delta_k$-th frame going backward. 
    Then, the new depth predictions will be scale-shift-aligned to the previous frames based on the 
    $T_k$ overlapping frames. We use $N = 32$, $T_0 = 8$, $T_k = 2$, $\Delta_k = 12$.}
    \label{fig:long_inference}
\end{figure}

To handle videos of arbitrary length, a straightforward approach is simply to concatenate the model outputs 
from different video windows. However, this method fails to ensure smooth transitions between windows. A 
more sophisticated technique entails inferring video windows with overlapping regions. By utilizing the 
predicted depth of the overlapping regions to compute an affine transform, predictions from one window can 
be aligned with those from another. Nevertheless, this method can introduce accumulated errors through 
successive affine alignments, leading to drift in extended videos. To address these challenges in ultra-long 
videos with a limited inference window size, we proposed key-frame referencing to inherit scale and shift 
information from past predictions and overlapping interpolation to ensure smooth inference across local windows.

\textbf{Key-frame referencing.}
As illustrated in Fig.~\ref{fig:long_inference}, a subsequent video clip for inference is composed of 
three parts: $N - T_0 - T_k$ future frames, $T_0$ overlapping frames from the previous clip, and $T_k$ key 
frames. The key frames are subsampled from the previous frames with an interval of size $\Delta_k$. 
Therefore, the video clip to be consumed shares the same length as during training. This approach 
incorporates content from earlier windows into the current window with minimal computation burden. 
Furthermore, we carefully select the values of $T_k$ and $\Delta_k$ to ensure that the first frame of a 
video is always positioned at the beginning of each clip, thereby enhancing depth consistency for extended 
videos. According to our experimental results, such simple strategy can significantly reduce accumulated 
scale drift, especially for long video.

\textbf{Depth clip stitching.}
Using $T_0$ overlapping frames (in Fig.~\ref{fig:long_inference}) between two consecutive windows is crucial 
for avoiding flickering depth predictions. The effects of overlapping frames are twofold. First, by sharing 
partial frame features, the depth content across consecutive windows will be more similar. Second, the depth 
prediction for the overlapping frames is updated by interpolating between the two segments. Assume the depth 
for the $o_i$-th overlapping frame from the previous window is denoted by $D^{\mathrm{pre}}_{o_i}$ and the 
current segment’s depth is denoted by $D^{\mathrm{cur}}_{o_i}$. The final depth is updated as

\[
D_{o_i} = D^{\mathrm{pre}}_{o_i} \cdot \frac{i}{T_0}
          + D^{\mathrm{cur}}_{o_i} \cdot \left( 1 - \frac{i}{T_0} \right),
\]

where $i$ ranges from $1$ to $T_0$.


\end{document}
