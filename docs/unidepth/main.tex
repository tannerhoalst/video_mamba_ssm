\documentclass[lettersize,journal]{IEEEtran}
\usepackage{svg}

% \usepackage{pgfplots}
% \pgfplotsset{compat=newest}
% \usepackage{tikz}
\usepackage{textcomp}
\usepackage{stfloats}


% *** CITATION PACKAGES ***
\usepackage{cite}


% *** MATH PACKAGES ***
\usepackage{graphicx}
\usepackage{amsmath}
\usepackage{amssymb}
\usepackage{algorithm}
\usepackage{mathtools}


% *** SUBFIGURE PACKAGES ***
\usepackage[caption=false,font=normalsize,labelfont=sf,textfont=sf]{subfig}



% *** PDF, URL AND HYPERLINK PACKAGES ***
%
% \usepackage{url}
% url.sty was written by Donald Arseneau. It provides better support for
% handling and breaking URLs. url.sty is already installed on most LaTeX
% systems. The latest version and documentation can be obtained at:
% http://www.ctan.org/pkg/url
% Basically, \url{my_url_here}.

\usepackage{times}
\usepackage{epsfig}

% Include other packages here, before hyperref.
% \usepackage[utf8x]{inputenc}
\usepackage{color}
\usepackage{xcolor}
\usepackage{booktabs,siunitx} % better tables
\usepackage{makecell}
\usepackage{enumitem}
\usepackage[percent]{overpic}
\usepackage[caption=false]{subfig}
\usepackage{rotating}
\usepackage[misc]{ifsym}
\usepackage{xspace}
\usepackage{pifont}
\usepackage{multirow}
\usepackage{bm}
\usepackage[accsupp]{axessibility} 
\usepackage{tabularx}
% \usepackage{subcaption}
\usepackage[noend]{algpseudocode}

% If you comment hyperref and then uncomment it, you should delete
% egpaper.aux before re-running latex.  (Or just hit 'q' on the first latex
% run, let it finish, and you should be clear).
\usepackage[pagebackref=true,breaklinks=true,colorlinks,bookmarks=false]{hyperref}
\usepackage[capitalize]{cleveref}

\Crefname{section}{Section}{Sections}
\Crefname{table}{Table}{Tables}
\crefname{section}{Sec.}{Secs.}
\crefname{table}{Tab.}{Tabs.}

% New/updated commands.

% Math.
\if@mathematic
   \def\vec#1{\ensuremath{\mathchoice
                     {\mbox{\boldmath$\displaystyle\mathbf{#1}$}}
                     {\mbox{\boldmath$\textstyle\mathbf{#1}$}}
                     {\mbox{\boldmath$\scriptstyle\mathbf{#1}$}}
                     {\mbox{\boldmath$\scriptscriptstyle\mathbf{#1}$}}}}
\else
   \def\vec#1{\ensuremath{\mathchoice
                     {\mbox{\boldmath$\displaystyle#1$}}
                     {\mbox{\boldmath$\textstyle#1$}}
                     {\mbox{\boldmath$\scriptstyle#1$}}
                     {\mbox{\boldmath$\scriptscriptstyle#1$}}}}
\fi

% Detectable updates with colored text.
\newcommand{\blue}[1]{\textcolor{black}{#1}}

% Math.
\newcommand{\vect}[1]{\mathbf{#1}}
\newcommand{\vecnorm}[1]{\left\|#1\right\|}
\newtheorem{thm}{Theorem}
\newtheorem{proposition}{Proposition}[section]
\newtheorem{definition}{Definition}

% Language
\newcommand{\etal}{\textit{et al}.\ }
\newcommand{\ie}{\textit{i}.\textit{e}.\ }
\newcommand{\eg}{\textit{e}.\textit{g}.\ }
\newcommand{\wrt}{\textit{w}.\textit{r}.\textit{t}.\ }
\newcommand{\vs}{\textit{vs}.\ }

% Tables.
\newcommand\ver[1]{\rotatebox[origin=c]{90}{#1}}
\newcommand{\fl}[1]{\multicolumn{1}{c}{#1}}
\newcommand{\yes}{\checkmark}
\newcommand{\no}{$\times$}

% Best results.
\newcommand{\best}[1]{\mathbf{#1}}
\newcommand{\scnd}[1]{\underline{#1}}

% Figures.
\newcommand{\fnn}[1]{$\scriptstyle^{#1}$}

% Paragraphs instead of subsections.
\newcommand{\PAR}[1]{\vskip4pt \noindent{\bf #1~}}

% Names.
\newcommand*{\ourmodel}{UniDepthV2\@\xspace}
\newcommand{\cmark}{\ding{51}}
\newcommand{\xmark}{\ding{55}}


% Hyphenation.
\hyphenation{op-tical net-works semi-conduc-tor IEEE-Xplore}


\begin{document}

\title{UniDepthV2:\\Universal Monocular Metric Depth Estimation\\Made Simpler}


\author{
Luigi~Piccinelli,
Christos~Sakaridis,
Yung-Hsu~Yang,
Mattia~Segu,
Siyuan~Li,
Wim~Abbeloos,
and~Luc~Van~Gool% <-this % stops a space
% note need leading \protect in front of \\ to get a newline within \thanks as
% \\ is fragile and will error, could use \hfil\break instead.
\IEEEcompsocitemizethanks{
\IEEEcompsocthanksitem L.~Piccinelli, C.~Sakaridis, Y-H..~Yang, M.~Segu, and S.~Li are with ETH Z\"urich, Switzerland.
\IEEEcompsocthanksitem W.~Abbeloos is with Toyota Motor Europe, Belgium.
\IEEEcompsocthanksitem L.~Van Gool is with ETH Z\"urich, Switzerland, and with INSAIT, Sofia University, Bulgaria.}% <-this % stops an unwanted space
}

% note the % following the last \IEEEmembership and also \thanks - 
% these prevent an unwanted space from occurring between the last author name
% and the end of the author line. i.e., if you had this:
% 
% \author{....lastname \thanks{...} \thanks{...} }
%                     ^------------^------------^----Do not want these spaces!
%
% a space would be appended to the last name and could cause every name on that
% line to be shifted left slightly. This is one of those "LaTeX things". For
% instance, "\textbf{A} \textbf{B}" will typeset as "A B" not "AB". To get
% "AB" then you have to do: "\textbf{A}\textbf{B}"
% \thanks is no different in this regard, so shield the last } of each \thanks
% that ends a line with a % and do not let a space in before the next \thanks.
% Spaces after \IEEEmembership other than the last one are OK (and needed) as
% you are supposed to have spaces between the names. For what it is worth,
% this is a minor point as most people would not even notice if the said evil
% space somehow managed to creep in.



% The paper headers
% \markboth{IEEE Transactions on Pattern Analysis and Machine Intelligence,~Vol.~xx, No.~xx, August~2025}%
% {Piccinelli \MakeLowercase{\textit{et al.}}: UniDepthV2: Universal Metric Monocular Depth Estimation Made Simpler}

% \IEEEpubid{0000--0000/00\$00.00~\copyright~2021 IEEE}
% Remember, if you use this you must call \IEEEpubidadjcol in the second
% column for its text to clear the IEEEpubid mark.

\maketitle

\begin{abstract}
    Accurate monocular metric depth estimation (MMDE) is crucial to solving downstream tasks in 3D perception and modeling.
    However, the remarkable accuracy of recent MMDE methods is confined to their training domains.
    These methods fail to generalize to unseen domains even in the presence of moderate domain gaps, which hinders their practical applicability.
    We propose a new model, \ourmodel, capable of reconstructing metric 3D scenes from solely single images across domains.
    Departing from the existing MMDE paradigm, \ourmodel directly predicts metric 3D points from the input image at inference time without any additional information, striving for a universal and flexible MMDE solution.
    In particular, \ourmodel implements a self-promptable camera module predicting a dense camera representation to condition depth features.
    Our model exploits a pseudo-spherical output representation, which disentangles the camera and depth representations.
    In addition, we propose a geometric invariance loss that promotes the invariance of camera-prompted depth features.
    \ourmodel improves its predecessor UniDepth model via a new edge-guided loss which enhances the localization and sharpness of edges in the metric depth outputs, a revisited, simplified and more efficient architectural design, and an additional uncertainty-level output which enables downstream tasks requiring confidence.
    Thorough evaluations on ten depth datasets in a zero-shot regime consistently demonstrate the superior performance and generalization of \ourmodel.
    Code and models are available at: \href{https://github.com/lpiccinelli-eth/unidepth}{github.com/lpiccinelli-eth/UniDepth}.
\end{abstract}
    
\begin{IEEEkeywords}
    Depth estimation, 3D estimation, camera prediction, geometric perception, foundation model.
\end{IEEEkeywords}


\section{Introduction}
\label{sec:intro}

\IEEEPARstart{P}{recise} pixel-wise depth estimation is crucial to understanding the geometric scene structure, with applications in 3D modeling~\cite{deng2022nerf}, robotics~\cite{Zhou2019, dong2022depth4robotics}, and autonomous vehicles~\cite{wang2019depth4vehicles, park2021dd3d}.
However, delivering reliable metric scaled depth outputs is necessary to perform 3D reconstruction effectively, thus motivating the challenging and inherently ill-posed task of Monocular Metric Depth Estimation (MMDE).

\begin{figure}[ht]
    \centering
    \includegraphics[width=1.0\linewidth]{figures/assets/teaser1.pdf}
    \vspace{-2em}
    \caption{
    We introduce \ourmodel, a novel approach that directly predicts 3D points in a scene with only one image as input. 
    \ourmodel incorporates a camera self-prompting mechanism and leverages a spherical 3D output space defined by azimuth and elevation angles, and depth($\theta$, $\phi$, $z$).
    This design effectively separates camera and depth optimization by avoiding gradient flowing to the camera module due to depth-related error ($\varepsilon_z$) compared to the standard Cartesian representation.}
    \label{fig:teaser}
    \vspace{-1em}
\end{figure}



While existing MMDE methods~\cite{Eigen2014, Fu2018Dorn, Bhat2020adabins, Ranftl2021dpt, Patil2022p3depth, Yuan2022newcrf, piccinelli2023idisc} have demonstrated remarkable accuracy across different benchmarks, they require training and testing on datasets with similar camera intrinsics and scene scales.
Moreover, the training datasets typically have a limited size and contain little diversity in scenes and cameras.
These characteristics result in poor generalization to real-world inference scenarios~\cite{Wang2020traingermany}, where images are captured in uncontrolled, arbitrarily structured environments and cameras with arbitrary intrinsics. \blue{What makes the situation even worse is the imperfect nature of actual ground-truth depth which is used to supervise MMDE models, namely its sparsity and its incompleteness near edges, which results in blurry predictions with inaccurate fine-grained geometric details.}


Only a few methods~\cite{yin2023metric3d, guizilini2023zerodepth, hu2024metric3dv2} have addressed the challenging task of generalizable MMDE.
However, these methods assume controlled setups at test time, including camera intrinsics. 
While this assumption simplifies the task, it has two notable drawbacks.
Firstly, it does not address the full application spectrum, \eg in-the-wild video processing and crowd-sourced image analysis.
Secondly, the inherent camera parameter noise is directly injected into the model, leading to large inaccuracies in the high-noise case.
% \blue{At the same time, while there has recently been significant progress in improving the sharpness and localization of depth discontinuities and fine geometric details in monocular \emph{relative} depth estimation~\cite{ke2024marigold, dav}, these advances have not been extended to standard non-iterative \emph{metric} depth estimators that run a single feed-forward pass for inference and need to solve a more complex, scale-dependent task.}

In this work, we address the more demanding task of generalizable MMDE \emph{without} any reliance on additional external information, such as camera parameters, thus defining the universal MMDE task.
Our approach, named \ourmodel{}, extends UniDepth~\cite{piccinelli2024unidepth} and is the first that attempts to solve this challenging task without restrictions on scene composition and setup and distinguishes itself through its general and adaptable nature. 
Unlike existing methods, \ourmodel delivers metric 3D predictions for any scene \emph{solely} from a single image, waiving the need for extra information about scene or camera.
Furthermore, \ourmodel flexibly allows for the incorporation of additional camera information at test time. \blue{Simultaneously, \ourmodel achieves sharper depth predictions with better-localized depth discontinuities than the original UniDepth model thanks to a novel edge-guided loss that enhances the consistency of the local structure of depth predictions around edges with the respective structure in the ground truth.}

\blue{The design of \ourmodel} introduces a camera module that outputs a non-parametric, \ie{}dense camera representation, serving as the prompt to the depth module. 
However, relying only on this single additional module clearly results in challenges related to training stability and scale ambiguity.
We propose an effective pseudo-spherical representation of the output space to disentangle the camera and depth dimensions of this space.
This representation employs azimuth and elevation angle components for the camera and a radial component for the depth, forming a perfect orthogonal space between the camera plane and the depth axis.
% 
Moreover, \blue{the pinhole-based camera representation is positionally encoded via a sine encoding in \ourmodel, leading to a substantially more efficient computation compared to the spherical harmonic encoding of the pinhole-based representation of the original UniDepth.}
Figure~\ref{fig:teaser} depicts our camera self-prompting mechanism and the output space.
Additionally, we introduce a geometric invariance loss to enhance the robustness of depth estimation. 
The underlying idea is that the camera-conditioned depth \blue{outputs} from two views of the same image should exhibit reciprocal consistency.
In particular, we sample two geometric augmentations, creating different views for each training image, thus simulating different apparent cameras for the original scene. \blue{Besides the aforementioned consistency-oriented invariance loss, \ourmodel features an additional uncertainty output and respective loss. These pixel-level uncertainties are supervised with the differences between the respective depth predictions and their corresponding ground-truth values, and enable the utilization of our MMDE model in downstream tasks such as control which require confidence-aware perception inputs~\cite{bonzanini2021perception,mesbah2016stochastic,yang2023safe,bemporad2007robust} for certifiability.}

\blue{The overall contributions of the present, extended journal version of our work are the first universal MMDE methods, the original UniDepth and the newer \ourmodel,} which predict a point in metric 3D space for each pixel without \emph{any} input other than a single image. \blue{An earlier version of this work has appeared in the Conference on Computer Vision and Pattern Recognition~\cite{piccinelli2024unidepth} and has introduced our original UniDepth model. In~\cite{piccinelli2024unidepth}, we have first designed} a promptable camera module, an architectural component that learns a dense camera representation and allows for non-parametric camera conditioning.
Second, we \blue{have proposed} a pseudo-spherical representation of the output space, thus solving the intertwined nature of camera and depth prediction.
In addition, we \blue{have introduced} a geometric invariance loss to disentangle the camera information from the underlying 3D geometry of the scene.
\blue{Moreover, in the conference version, we have extensively evaluated and compared UniDepth}
% and re-evaluated seven MMDE State-of-the-Art (SotA) methods
on ten different datasets in a fair and comparable zero-shot setup to lay the ground for \blue{our novel} generalized MMDE task.
Owing to its design, UniDepth consistently set the state of the art even compared with non-zero-shot methods, ranking first \blue{at the time of its appearance} in the competitive official KITTI Depth Prediction Benchmark.
\blue{Compared to the aforementioned conference version, this article makes the following additional contributions:
\begin{enumerate}
    \item A revisited architectural design of the camera-conditioned monocular metric depth estimator network, which makes \ourmodel simpler, substantially more efficient in computation time and parameters, and at the same time more accurate than UniDepth. This design upgrade pertains to the simplification of the connections between the Camera Module and the Depth Module of the network, the more economic sinusoidal embedding of the pinhole-based dense camera representations fed to the Depth Module that we newly adopt, the inclusion of multi-resolution features and convolutional layers in our depth decoder, and the application of the geometric invariance loss solely on output-space features.
    \item A novel edge-guided scale-shift-invariant loss, which is computed from the predicted and ground-truth depth maps around geometric edges of the input, encourages \ourmodel to preserve the local structure of the depth map better, and thus enhances the sharpness of depth outputs substantially compared to UniDepth even on camera and scene domains which are unseen during training.
    \item An improved practical training strategy that presents the network with a greater diversity of input image shapes and resolutions within each mini-batch and hence with a larger range of intrinsic parameters of the assumed pinhole camera model, leading to increased robustness to the specific input distribution during inference.
    \item An additional, uncertainty-level output, which requires no additional supervisory signal during training yet allows to quantify confidence during inference reliably and thus enables downstream applications to geometric perception, \eg{} control, which require confidence-aware depth inputs.
    % \item The methodological novelties introduced lead to improved performance of \ourmodel both in the standard metric depth estimation task and in the more complex metric 3D estimation task compared to UniDepth across a wide range of camera and scene domains, which is demonstrated through an extensive set of comparisons to the latest state-of-the-art methods as well as ablation studies on 10 widely used depth estimation benchmarks, both in the challenging zero-shot evaluation setting and in the practical supervised fine-tuning setting, and sets the \emph{new state of the art} in MMDE. In particular, \ourmodel ranks first among published methods in the competitive official public KITTI Depth Prediction Benchmark.
\end{enumerate}
The methodological novelties introduced lead to improved performance, robustness, and efficiency of \ourmodel compared to UniDepth across a wide range of camera and scene domains.
This is demonstrated through an extensive set of comparisons to the latest state-of-the-art methods as well as ablation studies on 10 depth estimation benchmarks, both in the challenging zero-shot evaluation setting and in the practical supervised fine-tuning setting.
\ourmodel sets the overall \emph{new state of the art} in MMDE and ranks first among published methods in the competitive official public KITTI Depth Prediction Benchmark.}

\section{Related Work}
\label{sec:relwork}

\begin{figure*}[ht]
    \centering
    \includegraphics[width=1.0\linewidth]{figures/assets/overview3.pdf}
    \vspace{-2em}
    \caption{\textbf{Model Architecture.} \ourmodel utilizes solely the input image to generate the 3D output ($\mathbf{O}$). It bootstraps a dense camera prediction ($\mathbf{C}$) from the Camera Module, injecting prior knowledge on scene scale into the Depth Module via a cross-attention layer per resolution, with 4 layers in total. The camera representation corresponds to azimuth and elevation angles. The geometric invariance loss ($\mathcal{L}_{\mathrm{con}}$) enforces consistency between geometric camera-aware output tensors from different geometric augmentations ($\mathcal{T}_1$, $\mathcal{T}_2$). The depth output ($\mathbf{Z}_{\log}$) \blue{is obtained through an FPN-based decoder that gradually upsamples the feature maps and injects multi-resolution information}. The final output is the concatenation of the camera and depth tensors ($\mathbf{C} || \mathbf{Z}_{\log}$), creating two independent optimization spaces for $\mathcal{L}_{\lambda MSE}$. \blue{The depth output is supervised with the proposed Edge-guided Normalized L1-loss $\mathcal{L}_{EG-SSI}$. In addition, \ourmodel computes a prediction uncertainty ($\mathbf{\Sigma}$) which is supervised with an L1-loss on the error in log space between predicted and ground-truth depth.}}
    \label{fig:results:overview}
    \vspace{-1em}
\end{figure*}


\PAR{Metric and Scale-Agnostic Depth Estimation.}
It is crucial to distinguish Monocular Metric Depth Estimation (MMDE) from scale-agnostic, namely up-to-a-scale, monocular depth estimation.
MMDE SotA approaches typically confine training and testing to the same domain.
However, challenges arise, such as overfitting to the training scenario leading to considerable performance drops in the presence of minor domain gaps,
% potential covariate shift issues, 
often overlooked in benchmarks like NYU-Depthv2~\cite{silberman2012nyu} (NYU) and KITTI~\cite{Geiger2012kitti}.
On the other hand, scale-agnostic depth methods, pioneered by MiDaS~\cite{ranftl2020midas}, OmniData~\cite{eftekhar2021omnidata}, and LeReS~\cite{yin2021leres}, show robust generalization by training on extensive datasets.
\blue{The paradigm has been elevated to another level by repurposing depth-conditioned generative methods for RGB to RGB-conditioned depth generative methods~\cite{ke2024marigold} or large-scale semi-supervised pre-training as in the DepthAnything series~\cite{yang2024da1, yang2024da2}.}
The limitation of all these methods lies in the absence of a metric output, hindering practical usage in downstream applications.

\PAR{Monocular Metric Depth Estimation.}
The introduction of end-to-end trainable neural networks in MMDE, pioneered by \cite{Eigen2014}, marked a significant milestone, also introducing the optimization process through the Scale-Invariant log loss ($\mathrm{SI}_{\log}$).
Subsequent developments witnessed the emergence of advanced networks, ranging from convolution-based architectures~\cite{Fu2018Dorn, Laina2016, Liu2015, Patil2022p3depth} to transformer-based approaches~\cite{Yang2021, Bhat2020adabins, Yuan2022newcrf, piccinelli2023idisc}.
Despite impressive achievements on established benchmarks, MMDE models face challenges in zero-shot scenarios, revealing the need for robust generalization against appearance and geometry domain shifts.

\PAR{General Monocular Metric Depth Estimation.}
Recent efforts focus on developing MMDE models~\cite{bhat2023zoedepth, guizilini2023zerodepth, yin2023metric3d} for general depth prediction across diverse domains.
These models often leverage camera awareness, either by directly incorporating external camera parameters into computations~\cite{facil2019camconvs, guizilini2023zerodepth} or by normalizing the shape or output depth based on intrinsic properties, as seen in~\cite{Lee2019bts, Lopez2020mapillary, yin2023metric3d, hu2024metric3dv2}.
\blue{A new paradigm recently emerged~\cite{piccinelli2024unidepth, bochkovskii2024depthpro}, where the goal is to directly estimate the 3D scene from the input image \emph{without any} additional information other than the RGB input.
Our approach fits in the latter new paradigm, namely universal MMDE: we do not require any additional prior information at test time, such as access to camera information.}



\section{Video Depth Anything}
\label{sec:method}

In this section, we introduce Video Depth Anything, a feed-forward video transformer model to efficiently estimate temporally consistent video depth. We adopt the affine-invariant depth, but share the same scale and shift across the entire video. The pipeline of our method is shown in Fig.~\ref{fig:overview_head}. Our model is built upon Depth Anything V2 with an additional temporal module and video dataset training (Sec.~\ref{subsec::video}). A novel loss to enfoce temporal consistency is proposed in Sec.~\ref{subsec::loss}. Finally, a strategy combined with overlapping frames and key frames is presented to efficiently support super-long video inference (Sect.~\ref{subsec::long}). 

\begin{figure*}[t]
  \centering
   \includegraphics[width=1.0\linewidth]{figures/imgs/overview_head_fix.pdf}

   \caption{\textbf{Overall pipeline and the spatio-temporal head}. Left: Our model is composed of a backbone encoder from Depth Anything V2 and a newly proposed spatio-temporal head. We jointly train our model on video data using ground-truth depth labels for supervision and on unlabeled images with pseudo labels generated by a teacher model. During training, only the head is learned. Right: Our spatiotemporal head inserts several temporal layers into the DPT head, while preserving the original structure of DPT head~\cite{ranftl2021vision}. 
   }%Right: Our spatio-temporal head incorporates a temporal layer with self-attention applied along the temporal axis into the DPT head~\cite{ranftl2021vision}.} 
   \label{fig:overview_head}
   \vspace{-18pt}
\end{figure*}
\subsection{Architecture}
\label{subsec::video}

Due to the lack of sufficient video depth data, we start with a pre-trained image depth estimation model, Depth Anything V2, and adopt a joint training strategy using both image and video data.

\noindent{\bf Depth Anything V2 Encoder.} Depth Anything V2~\cite{depth_anything_v2} is the current state-of-the-art monocular depth estimation model, characterized by its high accuracy and generalization capabilities. We use its trained model as our encoder. To reduce training costs and preserve well-learned features, the encoder is frozen during training.

Unlike monocular depth encoders that only accept image input, our training scenario requires the encoder to process simultaneously both video and image data. To extract features from video frames with an image encoder, we collapse the temporal dimension of a video clip into the batch dimension. The input data are denoted as $\mathbf{X} \in \mathbb{R}^{(B \times N) \times C \times H \times W} $, where $B$ represents the batch size, $N$ is the number of frames in the video clip, $N=1$ for the image as input, $C,H,W$ are the number of channels, height, width of the frames, respectively. The encoder takes $\mathbf{X}$ as input to produce a series of intermediate feature maps $\mathbf{F_i} \in \mathbb{R}^{(B \times N) \times (\frac{H}{p} \times \frac{W}{p}) \times C_i }$, $p$ is the patch size of the encoder. Although the image encoder extracts strong visual representations from individual frames, it neglects the temporal information interactions between frames. Thus, the spatiotemporal head is introduced to model the temporal relationship among the frames.


\noindent{\bf Spatiotemporal Head.}
The spatiotemporal head (STH) is built upon the DPT~\cite{ranftl2021vision} head and with the only modification being the insertion of temporal layers to capture temporal information.
%integrates temporal layers to capture temporal information. 
A temporal layer consists of a multi-head self-attention~\cite{vaswani2017attention} model (SA) and a feed-forward network (FFN). When inputting a feature $\mathbf{F_i}$ into the temporal layer, the temporal dimension $N$ is isolated, and self-attention is executed solely along the temporal dimension to facilitate the interaction of temporal features. To capture temporal positional relationships among different frames, we utilize absolute positional embedding to encode temporal positional information from the video sequence.

The spatiotemporal head uniformly samples 4 feature maps from $\mathbf{F_i}$ (including the final features from the encoder, denoted as $\mathbf{F_{4}}$) as inputs, and predicts a depth map $\mathbf{D} \in \mathbb{R}^{H \times W}$. As shown in Figure~\ref{fig:overview_head}, the selected features $\mathbf{F_i}$ are fed into the Reassemble layer to produce a feature pyramid. Then, the features are gradually fused from low resolution to high resolution by the Fusion layer. The Reassemble and Fusion layer are proposed by DPT~\cite{ranftl2021vision}. The final fused high-resolution feature maps are passed through the output layer to produce the depth map $\mathbf{D}$. To reduce the additional computational load, we insert the temporal layer at a few positions with lower feature resolutions. 

\subsection{Temporal Gradient Matching loss}
\label{subsec::loss}

In this section, we start with the Optical Flow Based Warping (OPW) loss, then explore new loss designs and ultimately propose a Temporal Gradient Matching Loss (TGM) that does not rely on optical flow, yet still ensures the temporal consistency of predictions between frames.

\noindent{\bf OPW loss.} To constrain temporal consistency, previous video models such as~\cite{wang2023neural,10.1145/3591106.3592264,Wang_2022} assume that the depths at corresponding positions in adjacent frames, identified through optical flow, are consistent, \textit{e.g.}, the Optical Flow based Warping (OPW) loss proposed in NVDS~\cite{wang2023neural}. OPW loss is computed after obtaining corresponding points on the basis of optical flow and warping. Specifically, for two consecutive depth prediction results, $p_i$ and $p_{i+1}$. $p_{i+1}$ is warped to $\hat{p_i}$ according to the wrapping relationship derived from the optical flow, and then the loss is calculated with:
\vspace{-3mm}
\begin{equation}
  \mathcal{L}_\text{OPW}= \frac {1}{N-1} \sum _ {i=2}^ {N} \parallel p_i - \hat{p_i}\parallel_1,
  \label{eq:opw_loss}
\vspace{-3mm} 
\end{equation}
where $N$ denotes the length of a video window, and $||\cdot||_1$ represents $\ell$1 distance. However, there is a fundamental issue with the OPW loss: the depth of corresponding points is not invariant across adjacent frames. This assumption holds true only when adjacent frames are stationary. For instance, in driving scenario, when a car is moving forward, the distance to static objects in front decreases relative to the car, violating the assumption of $\mathcal{L}_{OPW}$. To address this inherent issue of OPW, we propose a new loss function to constrain the temporal consistency of depth.

\noindent{\bf Temporal gradient matching loss (TGM)}. When calculating the loss, we do not assume that the depth of the corresponding points in adjacent frames remains unchanged. Instead, we posit that the change in depth of corresponding points between adjacent prediction frames should be consistent with the change observed in ground truth. We refer to this discrepancy as a stable error (SE) given by: 
\vspace{-3mm}
\begin{equation}
    \mathcal{L}_\text{SE} = \frac{1}{N-1} \sum_{i=1}^{N-1}\parallel \mid\hat{d_i}-d_{i}\mid - \mid\hat{g_i}-g_{i}\mid \parallel_1.
  \label{eq:stable_loss}
  \vspace{-3mm}
\end{equation}
Here, $d_i,g_i$ are scaled and shifted versions of the predictions and ground truth. $\hat{d_i},\hat{g_i}$ denotes the warped depth from the subsequent frame using optical flow. $\mid\cdot\mid$ is used to represent the absolute values. 

However, generating optical flow incurs additional overhead. To address the dependence on optical flow, we further generalize the above assumption. Specifically, it is not necessary to use the corresponding points obtained from the optical flow. Instead, we directly use the depth at the same coordinate in adjacent frames to calculate the loss. The assumption is that the change in depth at the same image position between adjacent frames should be consistent with that in the ground truth. Since this process is akin to calculating the gradient of values in temporal dimension, we name it Temporal Gradient Matching Loss, as given by
\vspace{-3mm}
\begin{equation}
    \mathcal{L}_\text{TGM} = \frac{1}{N-1} \sum_{i=1}^{N-1}\parallel \mid d_{i+1}-d_{i}\mid - \mid g_{i+1}-g_{i}\mid \parallel_1.
    \vspace{-3mm}
\end{equation}
In practice, we only compute the TGM loss in regions where the change in ground truth depth, \textit{i.e.}, $\mid g_{i + 1} - g_i \mid < 0.05 $. This threshold helps to avoid sudden changes in depth map caused by edges, dynamic objects, and other factors that introduce unsteadiness during training. 

Our total loss to supervise video depth data is as follows:
\vspace{-3mm}
\begin{equation}
    \mathcal{L}_{\text{all}} = \alpha \mathcal{L}_{\text{TGM}} + \beta  \mathcal{L}_{\text{ssi}},
  \label{eq:all_loss}
  \vspace{-3mm}
\end{equation}
where $\mathcal{L}_{ssi}$ is a scale- and shift-invariant loss to supervise single images proposed by MiDaS~\cite{birkl2023midas}. $\alpha$ and $\beta$ are weights to balance spatio-temporal consistency and spatial structure in a single frame. 

\subsection{Inference strategy for super-long sequence}
\label{subsec::long}

\begin{figure}
    \centering
    \includegraphics[width=\linewidth]{figures/imgs/long_inference-fix.pdf}
    \caption{\textbf{Inference strategy for long videos}. $N$ is the video clip lenght consumed by our model. Each inference video clip is built by $N-T_o-T_k$ future frames, $T_o$ overlapping/adjacent frames, and $T_k$ key frames. The key frames are selected by taking every $\Delta_k$-th frame going backward. Then, the new depth predictions will be scale-shift-aligned to the previous frames based on the $T_k$ overlapping frames. We use $N=32, T_o=8, T_k=2, \Delta_k=12$.}
    \label{fig:long_inference}
    \vspace{-18pt}
\end{figure}
To handle videos of arbitrary length, a straightforward approach is simply to concatenate the model outputs from different video windows. However, this method fails to ensure smooth transitions between windows. A more sophisticated technique entails inferring video windows with overlapping regions. By utilizing the predicted depth of the overlapping regions to compute an affine transform, predictions from one window can be aligned with those from another. Nevertheless, this method can introduce accumulated errors through successive affine alignments, leading to depth drift in extended videos. To address these challenges in ultra-long videos with a limited inference window size, we proposed key-frame referencing to inherit scale and shift information from past predictions and overlapping interpolation to ensure smooth inference across local windows.

\noindent{\bf Key-frame referencing.} As illustrated in ~\cref{fig:long_inference}, a subsequent video clip for inference is composed of three parts: $N-T_o-T_k$ future frames, $T_o$ overlapping frames from the previous clip and $T_k$ \textbf{key frames}. The key frames are subsampled from the previous frames with an interval of size $\Delta_k$. Therefore, the video clip to be consumed share the same length as during training. This approach incorporates content from earlier windows into the current window with minimal computation burden. Furthermore, we carefully select the values of $T_k$ and $\Delta_k$ to ensure that the first frame of a video is always positioned at the beginning of each clip, thereby enhancing depth consistency for extended videos. According to our experiment results, such simple strategy can significantly reduce accumulated scale drift, especially for long video.

\noindent{\bf Depth clip stitching.} Using $T_o$ overlapping frames (in ~\cref{fig:long_inference}) between two consecutive windows is crucial for avoiding flickering depth predictions. 
The effects of overlapping frames are twofold. 
First, by sharing partial frame features, the scale and shift across consecutive windows will be more similar. Second, the depth prediction for the overlapping frames is updated by interpolating between the two segments. Assume the depth for the $o_i$-th overlapping frame from the previous segment is denoted by $\mathbf{D}_{o_i}^{\text{pre}}$, and the depth from the current segment is denoted by $\mathbf{D}_{o_i}^{\text{cur}}$. The final depth is updated as $\mathbf{D}_{o_i} = \mathbf{D}_{o_i}^{\text{pre}} \cdot w_i + \mathbf{D}_{o_i}^{\text{cur}} \cdot (1 - w_i)$, where $w_i$ linearly decays from 1 to 0 as $i$ increases from 1 to $T_o$.

\section{Experiments}
\label{sec:experiments}


\subsection{Experimental Setup}
\label{ssec:experiments:setup}
\blue{\PAR{Data.} The training data is the combination of 24 publicly available datasets: A2D2~\cite{geyer2020a2d2}, Argoverse2~\cite{2021argoverse2}, ARKit-Scenes~\cite{baruch2021arkitscenes}, BEDLAM~\cite{black2023bedlam}, BlendedMVS~\cite{yao2020blendedmvs}, DL3DV~\cite{ling2024dl3dv}, DrivingStereo~\cite{yang2019drivingstereo}, DynamicReplica~\cite{karaev2023dynamicreplica}, EDEN~\cite{le2021eden}, HOI4D~\cite{liu2022hoi4d}, HM3D~\cite{ramakrishnan2021habitat}, Matterport3D~\cite{chang2017matterport3d}, Mapillary-PSD~\cite{Lopez2020mapillary}, MatrixCity~\cite{li2023matrixcity}, MegaDepth~\cite{li2018megadepth}, NianticMapFree~\cite{arnold2022mapfree}, PointOdyssey~\cite{zheng2023pointodyssey}, ScanNet~\cite{dai2017scannet}, ScanNet++~\cite{yeshwanthliu2023scannetpp}, TartanAir~\cite{wang2020tartanair}, Taskonomy~\cite{zamir2018taskonomy}, Waymo~\cite{sun2020waymo}, and WildRGBD~\cite{xia2024wildrgbd} for a total of 16M images.
We evaluate the generalizability of models by testing them on 8 datasets not seen during training, grouped in different domains that are defined based on indoor or outdoor settings. 
The indoor group corresponds to the validation splits of SUN-RGBD~\cite{Song2015sunrgbd}, IBims~\cite{koch2022ibims}, TUM-RGBD~\cite{sturm12tumrgbd}, and HAMMER~\cite{jung2022hammer}, while the outdoor group comprises ETH3D~\cite{schoeps2017eth3d}, Sintel~\cite{Butler2012sintel}, DDAD~\cite{Guizilini2020ddad}, and NuScenes~\cite{nuscenes}.}

\blue{\PAR{Evaluation Details.} All methods have been re-evaluated with a fair and consistent pipeline.
In particular, we do not exploit any test-time augmentations and we utilize the same weights for all zero-shot evaluations.
We use the checkpoint corresponding to the zero-shot model for each method, \ie not fine-tuned on KITTI or NYU.
The metrics utilized in the main experiments are $\mathrm{\delta_1^{SSI}}$, $\mathrm{F_{A}}$, and $\mathrm{\rho_{A}}$.
$\mathrm{\delta_1}$ measures the depth estimation performance.
$\mathrm{F_{A}}$ is the area under the curve (AUC) of F1-score~\cite{ornek20222metrics} up to $1/20$ of the datasets' maximum depth and evaluates 3D estimation accuracy.
$\mathrm{\rho_{A}}$ evaluates the camera performance and is the AUC of the average angular error of camera rays up to 15$^{\circ}$.
We do not use parametric evaluation of \eg{}focal length, since it is a less flexible metric across diverse camera models and perfectly unrectified images.
Moreover, we present the fine-tuning ability of \ourmodel by training the final checkpoint on KITTI and NYU-Depth V2 and evaluating in-domain, as per standard practice.}

\PAR{Implementation Details.} \ourmodel is implemented in PyTorch~\cite{pytorch} and CUDA~\cite{nickolls2008cuda}.
For training, we use the AdamW~\cite{Loshchilov2017adamw} optimizer ($\beta_1=0.9$, $\beta_2=0.999$) with an initial learning rate of $5\times{}10^{-5}$.
The learning rate is divided by a factor of 10 for the backbone weights for every experiment and weight decay is set to $0.1$.
We exploit Cosine Annealing as learning rate and weight decay scheduler to one-tenth starting from 30\% of the whole training.
\blue{We run 300k optimization iterations with a batch size of 128.
The training time amounts to 6 days on 16 NVIDIA 4090 with half precision.
The dataset sampling procedure follows a weighted sampler, where the weight of each dataset is its number of scenes.
Our augmentations are both geometric and photometric, \ie random resizing, cropping, and translation for the former type, and brightness, gamma, saturation, and hue shift for the latter.
We randomly sample the image ratio per batch between 2:1 and 1:2.}
Our ViT~\cite{Dosovitskiy2020VIT} backbone is initialized with weights from DINO-pre-trained~\cite{oquab2023dinov2} models.
For the ablations, we run 100k training steps with a ViT-S backbone, with the same training pipeline as for the main experiments.


\subsection{Comparison with The State of The Art}
\label{ssec:experiments:comparison}

\begin{table*}[t]
    \centering
    \caption{\textbf{Results for Indoor Domains.} All methods are tested in a zero-shot fashion. Missing values (\textcolor{gray}{-}) indicate the model's inability to produce the respective output. \dag: Requires ground-truth (GT) camera for 3D reconstruction. \ddag: Requires GT camera for 2D depth map inference.}
    \label{tab:results:indoor}
    \vspace{-1em}
    \resizebox{\linewidth}{!}{%
    \begin{tabular}{l|ccc|ccc|ccc|ccc}
    \toprule
    \multirow{2}{*}{\textbf{Method}} & \multicolumn{3}{c|}{SUNRGBD} & \multicolumn{3}{c|}{HAMMER} & \multicolumn{3}{c|}{IBims-1} & \multicolumn{3}{c}{TUM-RGBD} \\
     & $\mathrm{\delta_1}\uparrow$ & $\mathrm{F_A}\uparrow$ & $\mathrm{\rho_A}\uparrow$ & $\mathrm{\delta_1}\uparrow$ & $\mathrm{F_A}\uparrow$ & $\mathrm{\rho_A}\uparrow$ & $\mathrm{\delta_1}\uparrow$ & $\mathrm{F_A}\uparrow$ & $\mathrm{\rho_A}\uparrow$ & $\mathrm{\delta_1}\uparrow$ & $\mathrm{F_A}\uparrow$ & $\mathrm{\rho_A}\uparrow$ \\
    \midrule
    Metric3D\textsuperscript{\dag \ddag}~\cite{yin2023metric3d} & $1.9$ & \textcolor{gray}{-} & \textcolor{gray}{-} & $0.9$ & \textcolor{gray}{-} & \textcolor{gray}{-} & $75.1$ & \textcolor{gray}{-} & \textcolor{gray}{-} & $7.7$ & \textcolor{gray}{-} & \textcolor{gray}{-} \\
    Metric3Dv2\textsuperscript{\dag \ddag}~\cite{hu2024metric3dv2} & $81.2$ & \textcolor{gray}{-} & \textcolor{gray}{-} & $\best{65.3}$ & \textcolor{gray}{-} & \textcolor{gray}{-} & $68.4$ & \textcolor{gray}{-} & \textcolor{gray}{-} & $63.0$ & \textcolor{gray}{-} & \textcolor{gray}{-} \\
    ZoeDepth\textsuperscript{\dag}~\cite{bhat2023zoedepth} & $80.9$ & \textcolor{gray}{-} & \textcolor{gray}{-} & $0.9$ & \textcolor{gray}{-} & \textcolor{gray}{-} & $49.8$ & \textcolor{gray}{-} & \textcolor{gray}{-} & $55.6$ & \textcolor{gray}{-} & \textcolor{gray}{-} \\
    UniDepth~\cite{piccinelli2024unidepth} & $94.3$ & $78.6$ & $85.8$ & $1.8$ & $52.1$ & $55.3$ & $15.7$ & $30.3$ & $\best{76.6}$ & $72.3$ & $54.8$ & $86.8$ \\
    MASt3R~\cite{leroy2024master} & $80.1$ & $71.5$ & $\scnd{92.0}$ & $2.2$ & $38.1$ & $\best{86.5}$ & $61.0$ & $55.7$ & $76.0$ & $52.4$ & $44.1$ & $\scnd{93.7}$ \\
    DepthPro~\cite{bochkovskii2024depthpro} & $83.1$ & $71.1$ & $89.3$ & $29.4$ & $\scnd{71.0}$ & $69.1$ & $82.3$ & $62.8$ & $75.9$ & $56.9$ & $48.1$ & $\best{96.5}$ \\
    \midrule
    \ourmodel-Small & $90.8$ & $74.2$ & $87.7$ & $20.1$ & $52.6$ & $77.5$ & $86.6$ & $62.4$ & $67.5$ & $69.0$ & $50.6$ & $86.1$ \\
    \ourmodel-Base & $\scnd{94.4}$ & $\scnd{79.9}$ & $91.1$ & $30.6$ & $57.0$ & $65.6$ & $\scnd{89.7}$ & $\scnd{68.5}$ & $\scnd{76.5}$ & $\scnd{77.5}$ & $\scnd{57.3}$ & $89.4$ \\
    \ourmodel-Large & $\best{96.4}$ & $\best{84.6}$ & $\best{93.4}$ & $\scnd{64.5}$ & $\best{74.9}$ & $\scnd{78.3}$ & $\best{94.5}$ & $\best{70.9}$ & $74.1$ & $\best{90.5}$ & $\best{62.9}$ & $89.6$ \\
    \bottomrule
    \end{tabular}%
    }
\vspace{-1em}
\end{table*}
\begin{table*}[t]
    \centering
    \caption{\textbf{Results for Outdoor Domains.} All methods are tested in a zero-shot fashion. Missing values (\textcolor{gray}{-}) indicate the model's inability to produce the respective output. \dag: Requires ground-truth (GT) camera for 3D reconstruction. \ddag: Requires GT camera for 2D depth map inference.}
    \label{tab:results:outdoor}
    \vspace{-1em}
    \resizebox{\linewidth}{!}{%
    \begin{tabular}{l|ccc|ccc|ccc|ccc}
    \toprule
    \multirow{2}{*}{\textbf{Method}}  & \multicolumn{3}{c|}{ETH3D} & \multicolumn{3}{c|}{Sintel} & \multicolumn{3}{c|}{DDAD} & \multicolumn{3}{c}{NuScenes} \\
     & $\mathrm{\delta_1}\uparrow$ & $\mathrm{F_A}\uparrow$ & $\mathrm{\rho_A}\uparrow$ & $\mathrm{\delta_1}\uparrow$ & $\mathrm{F_A}\uparrow$ & $\mathrm{\rho_A}\uparrow$ & $\mathrm{\delta_1}\uparrow$ & $\mathrm{F_A}\uparrow$ & $\mathrm{\rho_A}\uparrow$ & $\mathrm{\delta_1}\uparrow$ & $\mathrm{F_A}\uparrow$ & $\mathrm{\rho_A}\uparrow$ \\
    \midrule
    Metric3D\textsuperscript{\dag \ddag}~\cite{yin2023metric3d} & $19.7$ & \textcolor{gray}{-} & \textcolor{gray}{-} & $1.4$ & \textcolor{gray}{-} & \textcolor{gray}{-} & $81.9$ & \textcolor{gray}{-} & \textcolor{gray}{-} & $75.4$ & \textcolor{gray}{-} & \textcolor{gray}{-} \\
    Metric3Dv2\textsuperscript{\dag \ddag}~\cite{hu2024metric3dv2} & $\best{90.0}$ & \textcolor{gray}{-} & \textcolor{gray}{-} & $\best{34.5}$ & \textcolor{gray}{-} & \textcolor{gray}{-} & $\scnd{87.6}$ & \textcolor{gray}{-} & \textcolor{gray}{-} & $84.1$ & \textcolor{gray}{-} & \textcolor{gray}{-} \\
    ZoeDepth\textsuperscript{\dag}~\cite{bhat2023zoedepth} & $33.8$ & \textcolor{gray}{-} & \textcolor{gray}{-} & $5.6$ & \textcolor{gray}{-} & \textcolor{gray}{-} & $27.9$ & \textcolor{gray}{-} & \textcolor{gray}{-} & $33.8$ & \textcolor{gray}{-} & \textcolor{gray}{-} \\
    UniDepth~\cite{piccinelli2024unidepth} & $18.5$ & $27.6$ & $42.6$ & $13.2$ & $40.2$ & $65.6$ & $85.8$ & $\scnd{72.8}$ & $\best{98.1}$ & $84.6$ & $\scnd{64.4}$ & $\best{97.7}$ \\
    MASt3R~\cite{leroy2024master} & $21.4$ & $28.4$ & $\scnd{92.2}$ & $17.2$ & $41.5$ & $72.2$ & $4.3$ & $22.1$ & $74.6$ & $2.7$ & $13.6$ & $78.3$ \\
    DepthPro~\cite{bochkovskii2024depthpro} & $39.7$ & $41.2$ & $77.4$ & $26.2$ & $49.7$ & $75.2$ & $29.9$ & $42.1$ & $83.0$ & $56.6$ & $46.5$ & $79.1$ \\
    \midrule
    \ourmodel-Small & $64.6$ & $44.3$ & $78.4$ & $14.6$ & $37.1$ & $73.5$ & $83.3$ & $68.5$ & $94.7$ & $82.1$ & $59.7$ & $96.2$ \\
    \ourmodel-Base & $75.4$ & $\scnd{53.5}$ & $91.4$ & $31.9$ & $\best{51.8}$ & $\scnd{75.9}$ & $86.8$ & $71.4$ & $96.1$ & $\scnd{85.3}$ & $63.6$ & $96.6$ \\
    \ourmodel-Large & $\scnd{85.2}$ & $\best{59.3}$ & $\best{92.6}$ & $\scnd{34.4}$ & $\scnd{51.4}$ & $\best{76.3}$ & $\best{88.2}$ & $\best{73.3}$ & $\scnd{96.7}$ & $\best{87.0}$ & $\best{66.7}$ & $\scnd{97.2}$ \\
    \bottomrule
    \end{tabular}%
    }
    \vspace{-1em}
\end{table*}
\begin{figure}[t]
    \centering
    \includegraphics[width=1.0\linewidth]{figures/assets/normalized_plot_resistance.pdf}
    \vspace{-2em}
    \caption{\textbf{Invariance to image shape.} \ourmodel is trained with a variable input shape pipeline in addition to random resizing for each of the image pairs. The proposed training strategy improves the robustness in terms of predicted depth scale and accuracy ($\delta_1$) to the input image's shape compared to two other state-of-the-art methods.}
    \label{fig:results:shape_invariance}
    \vspace{-1em}
\end{figure}



\blue{We evaluate our method on eight zero-shot validation sets, covering both indoor and outdoor scenes, as shown in \Cref{tab:results:indoor} and \Cref{tab:results:outdoor}, respectively. Our model performs better than or at least on par with all baselines, even outperforming methods that require ground-truth camera parameters at inference time, such as \cite{yin2023metric3d, hu2024metric3dv2}.
Notably, \ourmodel excels in 3D estimation, as reflected in the $\mathrm{F_A}$ metric, where it achieves a consistent improvement ranging from 0.5\% to 18.1\% over the second-best method. Additionally, it outperforms UniDepth~\cite{piccinelli2024unidepth} in nearly all cases, except for the $\mathrm{\rho_A}$ metric on IBims-1, DDAD, and NuScenes.
This demonstrates that our proposed version is a significant step forward in both performance and efficiency.
However, the camera parameter estimation ($\mathrm{\rho_A}$) sees only marginal improvements, indicating that the limited diversity of training cameras remains a challenge that could be addressed with additional camera-only training, as suggested in~\cite{bochkovskii2024depthpro}.
\Cref{tab:results:nyu_ft} and \Cref{tab:results:kitti_ft} show results for models fine-tuned on the NYU and KITTI training sets and evaluated on their respective validation splits, following standard protocols.
Fine-tuning performance serves as an indicator of a model's ability to specialize to specific downstream tasks and domains.
\ourmodel effectively adapts to new domains and outperforms methods that were pre-trained on large, diverse datasets before fine-tuning on NYU or KITTI, such as~\cite{bhat2023zoedepth, hu2024metric3dv2, yang2024da2},
This is particularly evident in the outdoor setting (KITTI), as shown in \Cref{tab:results:kitti_ft}.
As detailed in \Cref{ssec:method:design}, our training strategy incorporates variable image aspect ratios and resolutions within the same distributed batch.
Combined with camera conditioning and invariance learning, this approach enhances the model’s robustness to changes in input image shape.
\Cref{fig:results:shape_invariance} quantifies this effect: the y-axis represents normalized metric accuracy ($\mathrm{\delta}_1$ scaled by the method’s maximum value), while the x-axis varies the image shape.
The normalization ensures a consistent scale across models.
\ourmodel is almost invariant to image shape, demonstrating that it can effectively trade off resolution for speed without sacrificing accuracy, as clearly illustrated in \Cref{fig:results:shape_invariance}.}


\begin{table}[t]
    \centering
    \caption{\textbf{Comparison on NYU validation set.} All models are trained on NYU. The first 4 are trained only on NYU. The last 4 are fine-tuned on NYU.}
    \vspace{-1em}
    \label{tab:results:nyu_ft}
    \resizebox{\columnwidth}{!}{
    \begin{tabular}{l|ccc|ccc}
        \toprule
        \multirow{2}{*}{\textbf{Method}} & $\mathrm{\delta}_{1}$ & $\mathrm{\delta}_{2}$ & $\mathrm{\delta}_{3}$ & $\mathrm{A.Rel}$ & $\mathrm{RMS}$ & $\mathrm{Log}_{10}$\\
         & \multicolumn{3}{c|}{\textit{Higher is better}} & \multicolumn{3}{c}{\textit{Lower is better}}\\
        \toprule
        BTS~\cite{Lee2019bts} & $88.5$ & $97.8$ & $99.4$ & $10.9$ & $0.391$ & $0.046$\\
        AdaBins~\cite{Bhat2020adabins} & $90.1$ & $98.3$ & $99.6$ & $10.3$ & $0.365$ & $0.044$\\
        NeWCRF~\cite{Yuan2022newcrf} & $92.1$ & $99.1$ & $\scnd{99.8}$ & $9.56$ & $0.333$ & $0.040$\\
        iDisc~\cite{piccinelli2023idisc} & $93.8$ & $99.2$ & $\scnd{99.8}$ & $8.61$ & $0.313$ & $0.037$\\
        ZoeDepth~\cite{bhat2023zoedepth} & $95.2$ & $\scnd{99.5}$ & $\scnd{99.8}$ & $7.70$ & $0.278$ & $0.033$\\
        Metric3Dv2~\cite{hu2024metric3dv2} & $\best{98.9}$ & $\best{99.8}$ & $\best{100}$ & $\scnd{4.70}$ & $\scnd{0.183}$ & $\best{0.020}$\\
        DepthAnythingv2~\cite{yang2024da2} & $98.4$ & $\mathbf{99.8}$ & $\mathbf{100}$ & $5.60$ & $0.206$ & $\scnd{0.024}$\\
        \midrule 
        \ourmodel & $\scnd{98.8}$ & $\best{99.8}$ & $\best{100}$ & $\best{4.68}$ & $\best{0.180}$ & $\best{0.020}$\\ % 262250
        \bottomrule
    \end{tabular}}
    \vspace{-1em}
\end{table}

\begin{table}[t]
    \centering
    \caption{\textbf{Comparison on KITTI Eigen-split validation set.} All models are trained on KITTI Eigen-split training and tested on the corresponding validation split. The first 4 are trained only on KITTI. The last 4 are fine-tuned on KITTI.}
    \vspace{-1em}
    \label{tab:results:kitti_ft}
    \resizebox{\columnwidth}{!}{
    \begin{tabular}{l|ccc|ccc}
        \toprule
        \multirow{2}{*}{\textbf{Method}} & $\mathrm{\delta}_{1}$ & $\mathrm{\delta}_{2}$ & $\mathrm{\delta}_{3}$ & $\mathrm{A.Rel}$ & $\mathrm{RMS}$ & $\mathrm{RMS}_{\log}$\\
         & \multicolumn{3}{c|}{\textit{Higher is better}} & \multicolumn{3}{c}{\textit{Lower is better}}\\
        \toprule
        BTS~\cite{Lee2019bts} & $96.2$ & $99.4$ & $99.8$ & $5.63$ & $2.43$ & $0.089$\\
        AdaBins~\cite{Bhat2020adabins} & $96.3$ & $99.5$ & $99.8$ & $5.85$ & $2.38$ & $0.089$\\
        NeWCRF~\cite{Yuan2022newcrf} & $97.5$ & $\scnd{99.7}$ & $\scnd{99.9}$ & $5.20$ & $2.07$ & $0.078$\\
        iDisc~\cite{piccinelli2023idisc} & $97.5$ & $\scnd{99.7}$ & $\scnd{99.9}$ &$5.09$ & $2.07$ & $0.077$\\
        ZoeDepth~\cite{bhat2023zoedepth} & $96.5$ & $99.1$ & $99.4$ & $5.76$ & $2.39$ & $0.089$ \\
        Metric3Dv2~\cite{yin2023metric3d} & $\scnd{98.5}$ & $\best{99.8}$ & $\best{100}$ & $\scnd{4.40}$ & $1.99$ & $\scnd{0.064}$\\
        DepthAnythingv2~\cite{yang2024da2} & $98.3$ & $\best{99.8}$ & $\best{100}$ & $4.50$ & $\scnd{1.86}$ & $0.067$ \\
        \midrule
        \ourmodel & $\best{98.9}$ & $\best{99.8}$ & $\scnd{99.9}$ & $\best{3.73}$ & $\best{1.71}$ & $\best{0.061}$ \\ %263000
        \bottomrule
    \end{tabular}}
    \vspace{-1em}
\end{table}

\begin{figure}[t]
    \centering
    \includegraphics[width=1.0\linewidth]{figures/assets/normalized_plot_confidence.pdf}
    \vspace{-2em}
    \caption{\textbf{Confidence invariance.} The uncertainty output of \ourmodel represents the predicted error. The confidence is obtained as the inverse uncertainty and the output is evaluated by taking into account only the pixels with a confidence higher than the corresponding x-axis. Y-axis reports the normalized RMSE to have a consistent scale among different datasets, where normalization involves dividing the RMSE by the value with threshold 0, namely evaluating over all pixels.}
    \label{fig:results:confidence}
    \vspace{-1em}
\end{figure}



\subsection{Ablation Studies}
\label{ssec:experiments:ablations}

The importance of each new component introduced in \ourmodel in \cref{sec:method} is evaluated by ablating the method in \blue{Tables \ref{tab:results:ablations_arch}, \ref{tab:results:ablations_loss}, and \ref{tab:results:ablations_version}.}
All ablations exploit the predicted camera representation, if not stated otherwise.
\blue{\Cref{tab:results:ablations_arch} evaluates the impact of various architectural modifications compared to UniDepth~\cite{piccinelli2024unidepth}, analyzing their effects on both performance and efficiency.
\Cref{tab:results:ablations_loss} assesses the importance of the proposed loss function (\cref{ssec:method:egssi}) and examines the effect of applying the geometric invariance loss originally introduced in UniDepth~\cite{piccinelli2024unidepth} (\cref{ssec:method:consistency}) in different spaces.
The rationale behind our design choices is to maintain simplicity while maximizing effectiveness.
Additionally, in \Cref{tab:results:ablations_version} we analyze the role of camera conditioning and report results for the original UniDepth under the same training and evaluation setup as our method for a direct comparison.
The evaluation is based on four key metrics: $\mathrm{\delta}_1$, which measures metric depth accuracy; $\mathrm{SI}_{\log}$, which assesses scale-invariant scene geometry; $\mathrm{F_A}$, which captures the 3D estimation capability; and $\mathrm{\rho_A}$, which evaluates monocular camera parameter estimation.
All reported metrics correspond to the aggregated zero-shot performance across datasets, as detailed in \cref{ssec:experiments:setup}.}

\begin{figure*}[t]
    \renewcommand{\arraystretch}{1}
    \centering
    \small
    \hspace{-5pt}
    \begin{tabular}{cc|ccc}
        \multirow{1}{*}[0.5in]{\rotatebox[origin=c]{90}{KITTI}}
        & \includegraphics[width=0.22\linewidth]{figures/assets/edges/kitti_rgb.pdf}
        & \includegraphics[width=0.22\linewidth]{figures/assets/edges/kitti_v1.pdf}
        & \includegraphics[width=0.22\linewidth]{figures/assets/edges/kitti_no_loss.pdf}
        & \includegraphics[width=0.22\linewidth]{figures/assets/edges/kitti_v2.pdf}\\
        
        \multirow{1}{*}[0.5in]{\rotatebox[origin=c]{90}{Sintel}}
        & \includegraphics[width=0.22\linewidth]{figures/assets/edges/sintel_rgb.pdf}
        & \includegraphics[width=0.22\linewidth]{figures/assets/edges/sintel_v1.pdf}
        & \includegraphics[width=0.22\linewidth]{figures/assets/edges/sintel_no_loss.pdf}
        & \includegraphics[width=0.22\linewidth]{figures/assets/edges/sintel_v2.pdf}\\
        
        % \multirow{1}{*}[0.5in]{\rotatebox[origin=c]{90}{ETH3D}}
        % & \includegraphics[width=0.145\linewidth]{figures/assets/edges/eth_rgb.pdf}
        % & \includegraphics[width=0.145\linewidth]{figures/assets/edges/eth_v1.pdf}
        % & \includegraphics[width=0.145\linewidth]{figures/assets/edges/eth_no_loss.pdf}
        % & \includegraphics[width=0.145\linewidth]{figures/assets/edges/eth_v2.pdf} \\
        & RGB & UniDepth~\cite{piccinelli2024unidepth} & UniDepthV2 w/o $\mathcal{L}_\mathrm{EG-SSI}$ & \ourmodel\\
    \end{tabular}
    \vspace{-1em}
    \caption{\textbf{Comparisons of predicted edges.} Each row displays the input RGB image and the 2D depth maps predicted by compared methods, color-coded with the \textit{magma reverse} colormap with a range between 0 and 50 meters. Better viewed on a screen and zoomed in.
    }
    \label{fig:results:edges}
    \vspace{-1em}
\end{figure*}

\begin{table}[t]
    \centering
    \caption{\textbf{Architectural ablations.} The different architectural additions (``+'') and subtractions (``-'') from the original UniDepth~\cite{piccinelli2024unidepth} are reported. ``- SHE + Sine'': camera encoding via Sine encoding instead of Spherical Harmonic Transform of the pinhole-based pencil of rays. ``- Attention'': attention layers in the decoder are removed. ``+ ResNet Blocks'': the attention layers in the decoder are substituted with simpler ResNet blocks. ``+ Multi-resol.'': the decoder has lateral connections with the shallower encoder layer, rather than a simpler merging of all resolutions in the bottleneck.}
    \vspace{-1em}
    \label{tab:results:ablations_arch}
    \resizebox{\linewidth}{!}{%
    \begin{tabular}{ll|cccc|cc}
    \toprule
    & \multirow{2}{*}{\textbf{Architecture}} & \multicolumn{4}{c|}{{Performance}} & \multicolumn{2}{c}{{Efficiency}} \\
     & & $\mathrm{\delta_1}\uparrow$ & $\mathrm{SI_{\log}}\downarrow$ & $\mathrm{F_A}\uparrow$ & $\mathrm{\rho_A}\uparrow$ & Latency$\downarrow$ & Params$\downarrow$\\
    \midrule
    1 & UniDepth~\cite{piccinelli2024unidepth} & $54.5$ & $16.4$ & $56.1$ & $77.1$ & 73.2 & 35.2 \\
    2 & - SHE + Sine & $54.6$ & $16.4$ & $56.0$ & $76.9$ & 53.2 & 35.2 \\
    3 & - Attention & $50.3$ & $17.9$ & $51.0$ & $76.6$ & 20.4 & 29.0 \\
    4 & + ResNet Blocks & $52.6$ & $16.6$ & $55.0$ & $76.6$ & 24.0 & 33.5 \\
    5 & + Multi-resol. & $54.5$ & $16.3$ & $56.0$ & $77.9$ & 25.0 & 34.2 \\
    \bottomrule
    \end{tabular}%
    }
    \vspace{-1em}
\end{table}
\begin{table}[t]
    \centering
    \caption{\textbf{Loss ablations.} $\mathcal{L}_{\mathrm{EG-SSI}}$ refers to either employing or not the proposed Edge-Guided Normalized loss; $\mathbf{O}_{\mathcal{L}_\mathrm{con}}$ indicates the output there the geometry consistency loss is applied to.}
    \vspace{-1em}
    \label{tab:results:ablations_loss}
    \resizebox{\linewidth}{!}{%
    \begin{tabular}{lcc|cccc}
    \toprule
    & \multirow{2}{*}{$\mathcal{L}_{\mathrm{EG-SSI}}$} & \multirow{2}{*}{$\mathbf{O}_{\mathcal{L}_\mathrm{con}}$} & \multicolumn{4}{c}{{Zero-shot Test}}\\
     & & & $\mathrm{\delta_1}\uparrow$ & $\mathrm{SI_{\log}}\downarrow$ & $\mathrm{F_A}\uparrow$ & $\mathrm{\rho_A}\uparrow$\\
    \midrule
    1 & \xmark & $\mathbf{D|E}$ & $54.5$ & $16.3$ & $56.0$ & $77.9$\\
    2 & \xmark & $\mathbf{Z}$   & $55.3$ & $16.2$ & $56.1$ & $78.2$\\
    3 & \cmark & $\mathbf{Z}$   & $60.0$ & $15.3$ & $57.9$ & $79.8$\\
    \bottomrule
    \end{tabular}%
    }
\vspace{-1em}
\end{table}


\begin{table}[t]
    \centering
    \caption{\textbf{Model ablations.} The ``Model'' column refers to architecture and training strategy employed. ``V1'' is the original UniDepth, while ``V2'' is the proposed \ourmodel. ``Cond'' specifies whether the camera-prompting mechanism is present or not.}
    \vspace{-1em}
    \label{tab:results:ablations_version}
    \resizebox{\linewidth}{!}{%
    \begin{tabular}{lcc|cccc}
    \toprule
    & \multirow{2}{*}{\textbf{Model}} & \multirow{2}{*}{\textbf{Cond}} & \multicolumn{4}{c}{{Zero-shot Test}}\\
     & & & $\mathrm{\delta_1}\uparrow$ & $\mathrm{SI_{\log}}\downarrow$ & $\mathrm{F_A}\uparrow$ & $\mathrm{\rho_A}\uparrow$\\
    \midrule
    1 & V1 & \xmark & $50.1$ & $18.0$ & $50.8$ & $76.7$\\
    2 & V1 & \cmark & $54.5$ & $16.4$ & $56.1$ & $77.1$\\
    3 & V2 & \xmark & $49.3$ & $18.4$ & $49.2$ & $76.6$\\
    4 & V2 & \cmark & $54.5$ & $16.3$ & $56.0$ & $77.9$\\
    \bottomrule
    \end{tabular}%
    }
\vspace{-1em}
\end{table}

\blue{\PAR{Architecture.} \Cref{tab:results:ablations_arch} outlines the key modifications that transform the original UniDepth~\cite{piccinelli2024unidepth} architecture into \ourmodel.
The first major change is the removal of spherical harmonics (SH)-based encoding, which is computationally inefficient.
Instead, we revert to standard Sine encoding (row 2).
While the difference in performance is minimal in our setup, we hypothesize that the encoding’s impact diminishes as the model benefits from larger and more diverse training data across different cameras.
Next, we eliminate the attention mechanism in row 3 due to its high computational cost.
This removal results in a significant performance drop, \eg{}-4.3\% for $\mathrm{\delta}_1$, but yields a greater than 2x improvement in efficiency.
In row 4, we replace the pure MLP-based decoder with ResNet blocks, introducing spatial $3\times3$ convolutions.
This modification enhances performance by leveraging local spatial structure while inducing a minimal impact on efficiency.
Finally, row 5 integrates a multi-resolution feature fusion from the encoder to the decoder, following an FPN-style design.
This final architecture significantly reduces computational cost while preserving overall performance: the final model (row 5) achieves similar performance to the original UniDepth (row 1) while requiring only one-third of the computation.}
\blue{\PAR{$\mathcal{L}_{\mathrm{EG-SSI}}$ Loss.} The effectiveness of the proposed $\mathcal{L}_{\mathrm{EG-SSI}}$ loss, detailed in \cref{ssec:method:egssi}, is evaluated in row 2 \vs row 3 of \Cref{tab:results:ablations_loss}.
Introducing this loss results in a 4.7\% improvement in $\mathrm{\delta}_1$ and a 1.8\% improvement in $\mathrm{F_A}$, demonstrating its contribution to both metric accuracy and 3D estimation.
Interestingly, despite $\mathcal{L}_{\mathrm{EG-SSI}}$ not explicitly supervising camera parameter estimation, the $\mathrm{\rho_A}$ metric also shows improvement.
This suggests that the loss contributes to a less noisy training process, leading to better feature representations in the encoder.
A qualitative comparison of the impact of $\mathcal{L}_{\mathrm{EG-SSI}}$ is presented in \cref{fig:results:edges}.
The difference between the third and fourth columns highlights the visual impact of the proposed loss, particularly in refining depth discontinuities.
Additionally, the comparison between the second and third columns illustrates the combined effect of architectural changes and increased data diversity, showing improved reconstruction of finer details, such as body parts that were previously smoothed or missed.}
\blue{\PAR{$\mathcal{L}_{\mathrm{con}}$ Output Space.} \ourmodel introduces multiple instances of camera-conditioned depth features $\mathbf{D}|\mathbf{E}$, corresponding to different decoder resolutions, as described in \cref{ssec:method:design}.
This contrasts with the original UniDepth~\cite{piccinelli2024unidepth}, which relied on a single conditioning point.
Given this architectural shift, we argue that deep conditioning may not be optimal.
Features at different resolutions encode varying levels of abstraction, and enforcing deep conditioning introduces additional design freedom.
\Cref{tab:results:ablations_loss} investigates where to apply the consistency loss ($\mathcal{L}_{\mathrm{con}}$) from~\cite{piccinelli2024unidepth}: either directly in the output space ($\mathbf{Z}$, row 2) or within the camera-conditioned features at each scale ($\mathbf{D}|\mathbf{E}$, row 1).
The results indicate minimal differences from applying the loss directly in the output space. Therefore, based on Occam's razor, we adopt the simpler and more effective design from row 2 as the final approach.}
\blue{\PAR{Conditioning Impact.} As previously explored in~\cite{piccinelli2024unidepth}, we analyze the impact of our proposed camera conditioning in \Cref{tab:results:ablations_version}.
This ablation includes both UniDepth and \ourmodel under the same conditions—without $\mathcal{L}_{\mathrm{EG-SSI}}$ and without invariance applied to deep features ($\mathbf{D}|\mathbf{E}$).
The results show that conditioning has a even stronger positive effect for \ourmodel, as evidenced by comparing row 3 \vs row 4 against the comparison of row 1 \vs row 2.}
\blue{\PAR{Confidence.} The confidence measure introduced in \cref{ssec:method:design} is evaluated on three zero-shot datasets, as shown in \cref{fig:results:confidence}.
The y-axis represents the normalized $\mathrm{RMSE}$, computed as $\mathrm{RMSE}$ divided by its per-dataset value at $x = 0$, while the x-axis corresponds to the confidence quantile.
For each quantile, the evaluation considers only pixels whose confidence exceeds the given threshold.
Ideally, confidence should be negatively correlated with error: if the confidence estimate is reliable, higher-confidence regions should exhibit lower $\mathrm{RMSE}$.
More specifically, \cref{fig:results:confidence} validates how the predicted confidence of \ourmodel negatively correlates with the error, thus showing its reliability.}


\section{Conclusion}
\label{sec:conclusion}

\blue{We introduced \ourmodel, a universal monocular metric depth estimation model that enhances generalization across diverse domains without requiring camera parameters at test time.
By improving both the model architecture and introducing new loss functions in the training objective, \ourmodel achieves state-of-the-art performance while enhancing computational efficiency, as demonstrated through extensive zero-shot and fine-tuning evaluations.
Additionally, our training strategy enables a flexible trade-off between inference speed and detail preservation by allowing variable input resolutions at test time while maintaining global scale consistency.}

% use section* for acknowledgment
\section*{Acknowledgments}
This work is funded by Toyota Motor Europe via the research project TRACE-Z\"urich. Additional thanks to Lavinia Recchioni for her editing and unwavering support.

% \newpage

% trigger a \newpage just before the given reference
% number - used to balance the columns on the last page
% adjust value as needed - may need to be readjusted if
% the document is modified later
% \IEEEtriggeratref{8}
% The "triggered" command can be changed if desired:
%\IEEEtriggercmd{\enlargethispage{-5in}}

% References section.

\bibliographystyle{IEEEtran}
\bibliography{IEEEabrv,refs}

% biography section

\vspace{-2em}

\begin{IEEEbiography}[{\includegraphics[width=1in,clip,keepaspectratio]{images/luigi.pdf}}]{Luigi Piccinelli} is a Ph.D. candidate at ETH Z\"urich, Computer Vision Lab, supervised by Prof. Luc Van Gool and Dr. Christos Sakaridis.
His research focuses on 3D perception, particularly advancing generalization for ill-posed problems such as monocular depth estimation, both from single images and videos.
He has also explored tracking and domain adaptation.
He obtained his B.Sc. and M.Sc. degrees in Electrical Engineering from University of Bologna and ETH Zurich, respectively.
\end{IEEEbiography}

\vspace{-2em}

\begin{IEEEbiography}[{\includegraphics[width=1in,clip,keepaspectratio]{images/christos.pdf}}]{Christos Sakaridis} is a lecturer at ETH Z\"urich and a senior postdoctoral researcher at Computer Vision Lab of ETH Z\"urich.
The focus of his research is on semantic and geometric visual perception, involving multiple domains, visual conditions, and visual or non-visual modalities. Since 2021, he has been the Principal Engineer in TRACE Zurich, a large-scale project on computer vision for autonomous cars and robots.
He received the ETH Z\"urich Career Seed Award in 2022.
He obtained his PhD from ETH Z\"urich in 2021, having worked at Computer Vision Lab.
Before that, he received his MSc in Computer Science from ETH Z\"urich in 2016 and his Diploma in Electrical and Computer Engineering from National Technical University of Athens in 2014.
\end{IEEEbiography}

% \vspace{-2em}

\begin{IEEEbiography}[{\includegraphics[width=1in,clip,keepaspectratio]{images/roy.pdf}}]{Yung-Hsu Yang} is a Ph.D. student at ETH Z\"urich supervised by Prof. Marc Pollefeys. My research interests include scene understanding and 3D Object Detection and Tracking. He obtained my M.Sc. and B.Sc. degrees in Electrical Engineering at National Tsing Hua University supervised by Prof. Min Sun.
Previously, he worked with Dr. Samuel Rota Bulo and Dr. Peter Kontschieder in dense prediction tasks.
\end{IEEEbiography}

\vspace{-2em}

\begin{IEEEbiography}[{\includegraphics[width=1in,clip,keepaspectratio]{images/mattia.pdf}}]{Mattia Segu} is a Ph.D. candidate at the Computer Vision Lab at ETH Z\"urich, co-supervised by Prof. Luc Van Gool and Prof. Bernt Schiele as a member of the Max Planck ETH Center for Learning Systems.
His research focuses on advancing multiple object tracking methods that can learn end-to-end from long video sequences, adapt dynamically, and leverage limited annotations in a self-supervised fashion.
Currently, he is a student researcher at Google, contributing to Federico Tombari's team.
Additionally, he has worked on domain generalization, uncertainty estimation, and foundation models for object tracking and depth estimation.
\end{IEEEbiography}

\vspace{-2em}

\begin{IEEEbiography}[{\includegraphics[width=1in,clip,keepaspectratio]{images/siyuan.pdf}}]{Siyuan Li} is a Ph.D. student at the Computer Vision Laboratory, ETH Z\"urich, Switzerland, supervised by Dr. Martin Danelljan and Prof. Luc Van Gool.
His research focuses on computer vision and machine learning, with an emphasis on visual perception, open-world understanding, and multi-object tracking. He is particularly interested in developing scalable and generalizable models for autonomous driving and robotics.
His work has been published in top-tier computer vision conferences, including CVPR, ECCV, and ICCV.
\end{IEEEbiography}

\vspace{-2em}

\begin{IEEEbiography}[{\includegraphics[width=1in,clip,keepaspectratio]{images/wim.pdf}}]{Wim Abbeloos}
He earned an MSc in Applied Engineering from the University of Antwerp (2011) and then worked as a researcher and PhD student at both InViLab (University of Antwerp) and EAVISE (KU Leuven), focusing on 3D object detection, unsupervised 3D object discovery, and pose estimation for robotics. Subsequently, he joined Toyota Motor Europe (Belgium) in 2018, where he currently coordinates and manages research collaborations with top research institutes in Europe in the fields of computer vision and artificial intelligence, including automated driving and other application areas. Additionally, he supports the transfer and integration of the developed knowledge into future applications and products.
\end{IEEEbiography}

\vspace{-2em}

\begin{IEEEbiography}[{\includegraphics[width=1in,clip,keepaspectratio]{images/luc.pdf}}]{Luc Van Gool} is a full professor for Computer Vision at INSAIT and professor emeritus at ETH Z\"urich and the KU Leuven. He has authored over 900 papers.
He has been a program committee member of several major computer vision conferences (\eg ICCV’05, ICCV’11, and ECCV’14).
His main interests include 3D reconstruction and modeling, object recognition, and autonomous driving.
He received several best paper awards (\eg David Marr Prize ’98, Best Paper CVPR’07). He received the Koenderink Award in 2016 and the ``Distinguished Researcher'' nomination by the IEEE Computer Society in 2017.
In 2015 he also received the 5-yearly Excellence Prize by the Flemish Fund for Scientific Research. He was the holder of an ERC Advanced Grant (VarCity).
Currently, he leads computer vision research for autonomous driving in the context of the Toyota TRACE labs and has an extensive collaboration with Huawei on image and video enhancement.
\end{IEEEbiography}


% Can be used to pull up biographies so that the bottom of the last one
% is flush with the other column.
% \enlargethispage{-5in}



\end{document}