\documentclass{article}
\usepackage{graphicx} % Required for inserting images
\usepackage{float}

% Math packages
\usepackage{amsmath}        % for align, bmatrix, etc.
\usepackage{amsfonts}       % for \mathbb{}
\usepackage{amssymb}        % for extra math symbols

% For nicer text encoding (optional but recommended)
\usepackage[T1]{fontenc}
\usepackage[utf8]{inputenc}



\title{VideoMambaDepth}
\author{Tanner Hoalst}
\date{November 2025}

\begin{document}

\maketitle

\section{State-Space Models}

State Space Sequence Models (SSMs) represent a category of systems that transform a 
one-dimensional function or sequence $u(t)$ into $y(t)$. They are described by the following linear 
Ordinary Differential Equation (ODE):

\[
x'(t) = A x(t) + B u(t),
\tag{1}
\]

\[
y(t) = C x(t) + D u(t).
\tag{2}
\]

In this equation, $A \in \mathbb{R}^{N \times N}$ is the state matrix, and $B, C \in \mathbb{R}^N$ are its 
parameters and skip connection $D \in \mathbb{R}^1$, with $x(t) \in \mathbb{R}^N$ symbolizing the implicit 
latent state. SSMs possess several advantageous properties, such as linear complexity per time step and the 
ability for parallel computations, which aid in efficient training. Yet, standard SSMs generally require 
more memory compared to equivalent CNNs and face issues like vanishing gradients during training, limiting 
their widespread use in sequence modeling.

The evolution of SSMs led to the creation of Structured State Space Sequence Models (S4), which 
notably enhance the basic SSM framework. S4 achieves this by applying structured designs to the state matrix $A$ 
and incorporating an efficient algorithm. The state matrix in S4 is specifically developed and initialized 
using the High-Order Polynomial Projection Operator (HIPPO), facilitating the construction of deep 
sequence models that are both rich in capacity and adept at long-range reasoning. Remarkably, S4 has 
outperformed Transformers in the demanding Long Range Arena Benchmark.

Mamba represents a further advancement in SSMs, especially in discrete data modeling, such as text 
and genomic sequences. Mamba introduces two significant enhancements. Firstly, it incorporates an 
input-specific selection mechanism, differing from traditional, invariant SSMs. This mechanism filters 
information efficiently by customizing SSM parameters based on input data. Secondly, Mamba employs a 
hardware-optimized algorithm, which scales linearly with sequence length and uses a scanning process for 
recurrent computation, enhancing speed on contemporary hardware. Mamba's architecture, which combines SSM 
blocks with linear layers, is notably more streamlined. It has achieved top-tier results in various 
long-sequence fields, including language and genomics, demonstrating considerable computational efficiency 
in both training and inference phases.




\section{Self-Supervised Framework}

In this section we describe the framework of our model and describe how we provide the supervisory
signal during the training of our model. Fundamentally, our method is a form of Structure from Motion (SfM),
where the monocular camera is moving within a rigid environment to provide multiple views of that scene.
Our framework is built upon Monodepth2.

Let $I_t \in \mathbb{R}^{H \times W \times 3}$, $t \in \{-1, 0, 1\}$ be a frame in a monocular video sequence captured by a moving camera,
where $t$ is the frame index. Similarly, let $D_t \in \mathbb{R}^{H \times W}$ denote the depth map corresponding to image $I_t$.
The camera pose changes from time 0 to time $t$ ($t \in \{-1, 1\}$) is encoded by the $3 \times 3$ rotation matrix $R_t$ and the
translation vector $t_t$. We obtain the $4 \times 4$ camera transformation matrix thus:

\[
M_t = 
\begin{bmatrix}
R_t & t_t \\
0 & 1
\end{bmatrix}.
\tag{3}
\]

Our aim is to train two CNN networks to simultaneously estimate the pose of the camera, and the structure
of the scene, respectively:

\[
M = \theta_{\text{pose}}(I_t).
\tag{4}
\]

\[
D = \theta_{\text{depth}}(I_t).
\tag{5}
\]

Self-supervised depth prediction reformulates the learning task as a novel view-synthesis problem.
Specifically, during training, we let the coupled network synthesize the photo-consistency appearance of a
target frame from another viewpoint of the source frame. We treat the depth map as an intermediate variable
to constrain the network to complete the image synthesis task.

Let $(u, v) \in \mathbb{R}^2$ be the calibrated coordinates of a pixel in image $I_0$. In this case, let the
origin $(0, 0)$ be the top-left of the image. In the process of imaging, a 3D point $(X, Y, Z) \in \mathbb{R}^3$
projects onto $(u, v)$ through a perspective projection operator:

\[
\pi(X, Y, Z) = 
\left( 
f_x \frac{X}{Z} + c_x,\,
f_y \frac{Y}{Z} + c_y
\right)
= (u, v),
\tag{6}
\]

where $(f_x, f_y, c_x, c_y)$ are the camera intrinsic parameters. Therefore, given a depth map $D(u, v)$,
a 2D image point $(u, v)$ backprojects to a 3D point $(X, Y, Z)$ through backprojection:

\[
\pi^{-1}(u, v, D(u,v)) 
= D(u,v)
\left(
\frac{u - c_x}{f_x},\,
\frac{v - c_y}{f_y},\,
1
\right)
= (X, Y, Z).
\tag{7}
\]

Then the corresponding pixels in image $I_t$ can be computed as:

\[
(u', v') = \pi\!\left( M_t \cdot \pi^{-1}(u, v, D(u, v)) \right)
= g_{u,v}(I_t \mid D(u, v), M_t).
\tag{8}
\]

We project the pixels of an image to form a novel synthetic view (8). However, the projected coordinates
$(u', v')$ are continuous values. To obtain $I_t(u, v)$ we include a differentiable bilinear sampling
mechanism, as proposed in spatial transformer networks. We can now linearly interpolate the values
of the 4-pixel neighbors (top-left, top-right, bottom-left, bottom-right) of $(u', v')$ to give the RGB
intensities as follows:

\[
I^t(u,v) = \sum_{w} w^{uv} \, I^t(u', v'),
\tag{9}
\]

where $w^{uv}$ is linearly proportional to the spatial proximity between $(u, v)$ and $(u', v')$,
and $\sum_{w} w^{uv} = 1$.

\begin{figure}[H]
    \centering
    \includegraphics[width=\linewidth]{figures/scan_expand_merge.png}
    \caption{The scan expanding and scan merging operations in SS2D. 
    In the SS2D method, input patches follow distinct scanning paths. 
    Each sequence is independently processed by separate S6 blocks. 
    The results are merged to create a 2D feature map, which serves as the final output.}
    \label{fig:ss2d}
\end{figure}



\section{MambaDepth}

In self-supervised depth estimation, Monodepth2 and its derivatives, recognized for their 
symmetric encoder–decoder structure, are predominant. This structure is adept at extracting multi-level 
image features through conventional methods. However, the design is constrained in its ability to capture 
long-range dependencies in images, as the convolutional kernels focus on local areas. Each convolutional 
layer only processes features within its limited receptive field. Although skip connections aid in merging 
detailed and abstract features, they primarily enhance local feature combination, not the modeling of 
extensive range dependencies.

MambaDepth, a novel design, integrates the strengths of Monodepth2 and Mamba to 
comprehensively understand global contexts in self-supervised depth estimation. Figure 3 showcases the 
structure of MambaDepth, which is distinct in its composition, featuring an embedding layer, encoder, 
decoder, disparity heads, and straightforward skip connections, marking a departure from the classical 
designs often found in prior work.

In the initial stage, the embedding layer segments the input image, denoted as $x$ with dimensions 
$\mathbb{R}^{H \times W \times 3}$, into discrete $4 \times 4$ patches. These patches are then transformed 
to a predefined dimension $C$ (typically set to 96), resulting in a reshaped image representation, $x'$, with 
dimensions $\mathbb{R}^{\frac{H}{4} \times \frac{W}{4} \times C}$. $x'$ is normalized using Layer 
Normalization before its progression into the encoder for the extraction of features. The encoder itself is 
structured into four phases, incorporating our newly introduced feature fusion step at the conclusion of 
the first three phases to compact the spatial dimensions and amplify the channel depth, employing a 
configuration of $[2,2,2,2]$ MD blocks and channel dimensions scaling from $[C, 2C, 4C, 8C]$ through each 
phase.

Conversely, the decoder reverses this process over its four stages, using our introduced feature 
decomposition technique at the start of the final three stages to enlarge spatial dimensions and condense 
the channel count. Here, the arrangement of MD blocks is $[2,2,2,2]$, with channel dimensions inversely 
scaling from $[8C, 4C, 2C, C]$. The decoder culminates in 4 disparity heads that upscale the feature 
dimensions by a factor of four through feature decomposition, followed by a projection layer that adjusts 
the channel count to align with the target of self-supervised depth estimation, which is processed through 
a convolutional layer and a Sigmoid layer to generate the final depth map.

Skip connections within this architecture are implemented using a simple addition operation, purposely 
designed to avoid the incorporation of extra parameters, thus maintaining the model's efficiency and 
simplicity.



\section{MambaDepth Block}

At the heart of MambaDepth lies the MD (Mamba Depth) module, which is an adaptation from VMamba, 
as illustrated in Figure 5. This module begins with Layer Normalization of the input, which 
then bifurcates into two distinct pathways. The initial pathway channels the input through a linear 
transformation and an activation phase, whereas the second pathway subjects the input to a sequence 
involving a linear transformation, depthwise separable convolution, and activation, before directing it to 
the 2D-Selective-Scan (SS2D) component for advanced feature extraction. Following this, the extracted 
features undergo Layer Normalization, are then merged with the output of the first pathway through 
element-wise multiplication, and finally are integrated using a linear transformation. This process is 
augmented with a residual connection, culminating in the output of the MD block. For activation, the 
SiLU function is consistently utilized throughout this study.

The SS2D mechanism comprises three essential stages: an operation to expand the scan, an S6 block for 
processing, and a merging operation for the scans. As shown in Figure 4, the expansion phase unfolds the 
input image in four orientations (diagonally and anti-diagonally) into sequences. These sequences are then 
refined by the S6 block, a procedure that meticulously scans information from all directions to extract a 
comprehensive range of features. Following this, the sequences are recombined through a merging operation, 
ensuring the output image is resized back to the original dimensions. The innovative S6 block, evolving from 
Mamba and building upon the S4 structure, introduces a selective filter that dynamically adjusts to the 
input by fine-tuning the parameters of the State Space Model (SSM). This adjustment allows the system to 
selectively focus on and preserve relevant information, while discarding what is unnecessary.

\begin{figure}[H]
    \centering
    \includegraphics[width=\linewidth]{figures/mamba_depth_block.png}
    \caption{The detailed structure of the MD (MambaDepth) Block.}
    \label{fig:mambablock}
\end{figure}



\section{MambaDepth Block}

At the heart of MambaDepth lies the MD (Mamba Depth) module, which is an adaptation from VMamba, 
as illustrated in Figure 5. This module begins with Layer Normalization of the input, which 
then bifurcates into two distinct pathways. The initial pathway channels the input through a linear 
transformation and an activation phase, whereas the second pathway subjects the input to a sequence 
involving a linear transformation, depthwise separable convolution, and activation, before directing it to 
the 2D-Selective-Scan (SS2D) component for advanced feature extraction. Following this, the extracted 
features undergo Layer Normalization, are then merged with the output of the first pathway through 
element-wise multiplication, and finally are integrated using a linear transformation. This process is 
augmented with a residual connection, culminating in the output of the MD block. For activation, the 
SiLU function is consistently utilized throughout this study.

The SS2D mechanism comprises three essential stages: an operation to expand the scan, an S6 block for 
processing, and a merging operation for the scans. As shown in Figure 4, the expansion phase unfolds the 
input image in four orientations (diagonally and anti-diagonally) into sequences. These sequences are then 
refined by the S6 block, a procedure that meticulously scans information from all directions to extract a 
comprehensive range of features. Following this, the sequences are recombined through a merging operation, 
ensuring the output image is resized back to the original dimensions. The innovative S6 block, evolving from 
Mamba and building upon the S4 structure, introduces a selective filter that dynamically adjusts to the 
input by fine-tuning the parameters of the State Space Model (SSM). This adjustment allows the system to 
selectively focus on and preserve relevant information, while discarding what is unnecessary.



\section{Loss Function}

\textbf{Objective functions.}
In line with the methodologies described in, we adopt the conventional photometric loss
$p_{\ell}$, which is a combination of $L_1$ and SSIM losses:

\[
p_{\ell}(I_a, I_b)
= \frac{\alpha}{2}\big(1 - \text{SSIM}(I_a, I_b)\big)
+ (1 - \alpha)\lVert I_a - I_b \rVert_1.
\tag{10}
\]

To ensure proper depth regularization in areas lacking texture, we employ an edge-aware smooth loss,
applied in the following manner:

\[
L_s = 
|\partial_x d_t| e^{-|\partial_x I_t|}
+ 
|\partial_y d_t| e^{-|\partial_y I_t|}.
\tag{11}
\]

\textbf{Masking Strategy.}
In real-world settings, scenarios featuring stationary cameras and moving objects can disrupt the usual 
assumptions of a moving camera and static environment, negatively impacting the performance of 
self-supervised depth estimators. Previous studies have attempted to enhance prediction accuracy by 
incorporating a motion mask, which addresses moving objects using scene-specific instance segmentation 
models. However, this approach limits its applicability to new, unencountered scenarios. To maintain 
scalability, our method eschews the use of a motion mask for handling moving objects. Instead, we adopt 
the auto-masking strategy outlined in, which filters out static pixels and areas of low texture 
that appear unchanged between two consecutive frames in a sequence. This binary mask $\mu$ is calculated 
as per (12), employing the Iverson bracket notation:

\[
\mu = 
\big[ \min_{t'}\, p_{\ell}(I_t, I_{t' \rightarrow t}) 
< 
\min_{t'}\, p_{\ell}(I_t, I_{t'}) \big].
\tag{12}
\]

\textbf{Final Training Loss.}
We formulate the final loss by combining our per-pixel smooth loss with the masked photometric losses:

\[
L = \mu L_p + \lambda L_s.
\tag{13}
\]



\textbf{\section{Experements}}

\textbf{KITTI} dataset, known for its stereo image sequences, is widely utilized in 
self-supervised monocular depth estimation. We employ the Eigen split, comprising about 26{,}000 
images for training and 697 for testing. Our approach with MambaDepth involves training it from the 
beginning on KITTI under minimal conditions: it operates solely with auto-masking, without 
additional stereo pairs or auxiliary data. For testing purposes, we maintain a challenging scenario by 
using only a single frame as input, in contrast to other methods that might use multiple frames to enhance 
accuracy.

\textbf{Cityscapes} dataset, noted for its complexity and abundance of moving objects, serves as a 
testing ground to assess the adaptability of MambaDepth. To this end, we conduct a zero-shot evaluation 
on Cityscapes, utilizing a model pre-trained on KITTI. It is crucial to highlight that, unlike many 
competing approaches, we do not employ a motion mask in our evaluation. For data preparation, we follow 
the same preprocessing procedures outlined in, which are also adopted by other baselines, 
converting the image sequences into triplets.

\textbf{Make3D}. To assess the capability of MambaDepth to generalize to new, previously unseen 
images, the model, initially trained on the KITTI dataset, was subjected to a zero-shot evaluation using 
the Make3D dataset. Furthermore, supplementary visualizations of depth maps are provided.

\textbf{Implementation Details}

We developed our model using the PyTorch framework. It was trained on eight NVIDIA Tesla 
V100-SXM2 GPUs, with a batch size of 8. We pre-trained the architecture on the ImageNet-1k dataset, 
subsequently using these pre-trained weights to initialize both the encoder and decoder components of the 
model. In line with the approach in, we applied color and flip augmentations to the images 
during training. Both DepthNet and PoseNet were trained using the Adam Optimizer, with 
$\beta_1$ set at 0.9 and $\beta_2$ at 0.999, DepthNet being also trained concurrently. The learning rate 
starts at $1\text{e-}4$ and reduces to $1\text{e-}5$ after 15 epochs. We configured the SSIM weight at 
$\alpha = 0.85$ and the weight for the smooth loss term at $\lambda = 1\text{e-}3$.



\end{document}
